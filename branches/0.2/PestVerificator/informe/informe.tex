\documentclass{article}
\usepackage[latin1]{inputenc}

\author{Federico Bugni, Federico Paulovsky, Gervasio Perez}
\title{An�lisis Autom�tico de Programas}
\title{Trabajo pr�ctico N� 1: Verificaci�n de Programas}

\begin{document}

\maketitle

\section{Ejercicio 1}

TODO.

\section{Ejercicios 2 y 3}

En esta secci�n trataremos los aspectos del TP relacionados a la implementaci�n del verificador Pest.

\subsection{Consideraciones generales de dise�o}

En nuestro TP utilizamos extensivamente el patr�n \textit{Visitor} para las recorridas de los �rboles sint�cticos de programas Pest, de predicados l�gicos y de t�rminos.

\subsection{Desarrollo}

\subsubsection{Primer enfoque de verificaci�n}

Inicialmente optamos por implementar el verificador armando una f�rmula l�gica con cuantificadores, lo cual prob� ser arduo especialmente con respecto al \textit{debugging}.
Luego de la clase pr�ctica del Mi�rcoles 14/9 decidimos cambiar el enfoque completamente, convirtiendo el output CVC3 en una secuencia de declaraciones de variables, y comandos ASSERT y comandos QUERY para verificar condiciones. El cambio no result� problem�tico porque reci�n est�bamos llegando a implementar el if y buena parte del c�digo de los Visitors pudo ser reutilizada. De todas maneras

\subsubsection{Versi�n final}

Para el nuevo enfoque adoptamos el concepto de \textit{contexto} mencionado en clase. 

\subsection{Clases desarrolladas}

Para la realizaci�n del TP desarrollamos los paquetes \textbf{budapest.pest.pesttocvc3} y \textbf{budapest.pest.predtocvc3} que contienen el c�digo de traducci�n. �stos contienen las siguientes clases:
\begin{itemize}
\item PestToCVC3Translator: Implementaci�n de Visitor que recorre un programa Pest y lo traduce, devolviendo un String con los comandos CVC3 que lo verifican.
\item PestVarContext: Representaci�n de un contexto de variables Pest versionadas. 
\item PredVarReplacer y TrmVarReplacer: Visitors que realizan el reemplazo de variables en Predicados y en Terms respectivamente. Hacen uso de un PestVarContext de donde toman el nombre que debe usarse para cada variable en el reemplazo.
\item PredParamReplacer y TrmParamReplacer: Similares a los anteriores, pero hacen un reemplazo de un nombre de variable por un Term cualquiera. Necesarios para implementar el reemplazo de variables en una llamada a funci�n.
\item PestVarBinder: Auxiliar para calcular el mapeo de nombres de par�metro de un procedimiento a las expresiones que se usaron para invocarlo.
\end{itemize}

\subsubsection{PestToCVC3Translator}

Este Visitor hace uso de contextos de variables para resolver el nombre que debe tener cada "instancia" de una variable al momento de usarla en una f�rmula CVC.

\end{document}